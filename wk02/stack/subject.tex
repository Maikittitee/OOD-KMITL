ที่จอดรถของนาย ก เป็นส่วนที่แรเงาสีฟ้า ส่วนสีแดงเป็นที่ของนาย ข ซี่งเป็นญาติกัน ที่จอดรถของนาย ก และ นาย ข แคบมาก จอดรถได้เรียงเดี่ยว นาย ข ไม่ได้ใช้ที่จอดรถ แต่ อนุญาติให้นาย ก ใช้ที่จอดรถของเขาได้โดยไม่จอดรถแช่ไว้ เนื่องจากซอยแคบ ดังนั้นการมาจอด (arrive) และการรับรถ (depart)จะเป็นลักษณะของ stack เงื่อนไขคือ ในการรับรถ x ใดๆอยากให้ลำดับรถเป็นเหมือนเดิม ดังรูป simulate การจอดรถในที่จอดรถของนาย ก โดยใช้ operation ของ stack ข้างล่างเป็นตัวอย่าง output

การรับ input : รับ input 4 ค่าใน 1 บรรทัดโดยให้แยกโดย " " (space bar) โดยตำแหน่งแรกคือ จำนวนสูงสุดที่รถสามารถจอดได้ในซอยของ นาย ก ตำแหน่งที่สองคือ รถที่จอดอยู่ในซอยของ นาย ก ตำแหน่งที่สามคือ การกระทำเช่น ถ้าเป็น arrive จะทำการเพิ่มรถในซอย ส่วน depart จะทำการเอารถออกจากซอย โดยรถที่จะทำการเพิ่มหรือนำออกนั้นจะเป็น เลขในตำแหน่งที่ 4

***หมายเหตุ ถ้าในซอยไม่มีรถอยู่เลยให้ input = 0 ในตำแหน่งที่ 2***